\chapter{Coding-Guidelines \label{appendix_cg}}
\begin{verbatim}
- Jede Einrückungsebene wird mit einem Tabulator eingerückt.
- Nach Kommentarzeichen folgt ein Leerzeichen.
- Geschweifte Klammern werden direkt nach dem entsprechenden
  Statement geöffnet.
Bsp.:
if (a < b) {
    // Do something
} else if (a > b) {
    // Do another thing
} else {
    // Whatever
}

- Funktionen werden nach folgendem Schema benannt:
    modul_namensteil1_..._namensteilN();
Bsp.:
    io_obj_start();

- Defines und globale Variablen (keine Datei-globale, nur
  System-globale) werden GROß geschrieben.
- Jede Funktion muss einen Doxygen-Kommentar haben.
- Jede nicht-triviale Funktion muss im Code dokumentiert
  werden.
- Funktionen, die System-globale Variablen ändern müssen
  auf diesen Umstand explizit in der Dokumentation
  hinweisen.
- Parameter sind, wenn möglich, als const zu deklarieren.
- Lokale Variablen werden sinnvoll benannt und einzelne
  Namensteile durch _ getrennt.
Bsp.:
    uint8_t inbuf_start;
- Ein-Zeichen Variablen sollen nur für for-Schleifen
  verwendet werden.
- Der Typ der Variablen muss explizit angegeben werden,
  mit Bit-größe. (uint8_t, int8_t, uint16_t, int16_t, etc)
\end{verbatim}
