\chapter{Zusammenfassung und Ausblick}
Das Ergebnis dieser Arbeit zeigt, dass die Platine auf Befehle schnell und korrekt reagiert,
solange diese nicht gravierend von der Spezifikation abweichen. Es gehen keine Interrupts verloren, und die
Platine reagiert auf Änderungen in Bruchteilen einer Sekunde.\\
Dennoch gibt es einige Punkte, die aufgrund der begrenzten Zeit nicht getestet oder implementiert
werden konnten.\\
So ist beispielsweise das Verhalten des ABS abhängig von der Dauer der Hauptschleife. Normalerweise
ist dies kein Problem. Wenn allerdings Debug-Ausgaben vorgenommen werden, kann das ABS sich
auf eine Art und Weise verhalten, die nicht erwünscht ist (dauerndes Hin- und Herschwenken der Räder,
da kein Schleifendurchlauf während des Nullpunktes stattfindet). Um dies zu verhindern, kann man
einen Timer-Interrupt mit einer möglichst niedrigen Auflösung benutzen. Der Vorteil wäre hierbei,
dass das ABS nicht mehr abhängig von der Geschwindigkeit der Hauptschleife ist. Außerdem ist das
Abschalten des ABS gleichzusetzen mit dem Abschalten des zugehörigen Interrupts. Das wiederum
eliminiert den Overhead des ABS komplett, wenn dieses abgeschaltet ist. Der Overhead wird auf
ungefähr 2,5 \textmu{}s pro Schleifendurchlauf geschätzt. Der Nachteil besteht in einem geringen
Mehraufwand, da ein Funktionsaufruf durch eine Interrupt-Service-Routine ersetzt wird.\\
Noch eine Erweiterungsmöglichkeit für das System besteht in erhöhter Robustheit und zusätzlicher 
Fehlererkennung. So ist es möglich, eintreffende Befehle auf komplette syntaktische Korrektheit
zu überprüfen und nur solche Befehle zu akzeptieren, die diese Tests bestehen. Diese Erweiterung
muss allerdings möglichst effizient und einfach ausbaubar implementiert werden, um einerseits nicht
gegen die Designprinzipien des Systems zu handeln, andererseits das Laufzeitverhalten nicht
schwerwiegend zu beeinträchtigen. Zusätzlich zu dieser Fehlererkennung kann die Robustheit des Systems
durch ein Überwachungssystem erhöht werden, welches in periodischen Abständen, ermöglicht durch
die eingebauten Timer, die einzelnen Module des Gesamtsystems auf Anzeichen von Problemen untersucht,
wie z.B. volle Puffer, runaway-Befehle, Parser-Status-Korruption und dergleichen. Um die Möglichkeit
von runaway-Code auszuschließen, ist die Benutzung des eingebauten Watchdog-Timers möglich, dessen
timeout-Wert allerdings sehr sorgfältig gewählt werden muss. Der timeout-Wert darf nicht kleiner sein
als die längste Hauptschleifen-Iteration plus ein entsprechendes Sicherheitspolster.\\
Ein Bereich, der während der Arbeit gänzlich ausgeklammert wurde, ist die Möglichkeit, die
Motorplatine in einen Idle-Modus zu schicken. In diesem Modus verbraucht die Platine wesentlich weniger
Strom. Das würde die Nutzungsdauer der Batterie erhöhen, wenn die Praktikanten an solch einem Fahrzeug arbeiten
und die Motorplatine sich für längere Zeit im Leerlauf befindet. In dieser Hinsicht
wäre es auch interessant, den Idle-Modus -- bzw. das Verhalten des Idle-Modus -- durch Befehle während der
Laufzeit steuern zu können.\\
Wie im Kapitel über die Laufzeituntersuchung beschrieben wurde, ist die Datenrate über den I2C-Bus der größte
limitierende Faktor bei der Übermittlung von Fahrbefehlen von der Praktikumsplatine an die
Motorplatine. Damit dieses Problem minimiert wird, kann man zum einen erwägen, den I2C-Bus im
''high-speed''-Modus operieren zu lassen, oder eine eigene Punkt-zu-Punkt-Kommunikation mithilfe der freien Ports auf der
Motorplatine zu realisieren. Solch ein maßgeschneiderter Port mit einem eigens dafür entwickelten Protokoll
könnte die Latenz, die durch das Senden des Befehls entsteht, erheblich verringern.
