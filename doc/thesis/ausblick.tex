\chapter{Zusammenfassung und Ausblick}
% ABS berechenbarer gestalten, indem es während des 1ms int ausgeführt wird
% Eine zusätzliche Resetmöglichkeit einbauen, die nicht über I2C oder UART kommt (kabel an pin + abfragen alle 262 ms)
% Verbesserte Fehlererkennung
% Robusteres Verhalten bei fehlbefehlen und problematischen Situationen (buffer voll)
% Watchdog timer benutzen und auf eine gute Größe einstellen
% benutze Idle Mode, wenn ... idle
% Befehle zur Steuerung von Power-Mode
% kommunikation über eigenen Bus, kein I2C-Bus (könnte die Bandbreite beträchtlich erhöhen)
Am Ende dieser Arbeit ist die Platine in der Lage auf Befehle schnell und korrekt zu reagieren,
solange diese nicht schwerwiegend fehlgeformt sind. Es gehen keine Interrupts verloren und die
Platine reagiert auf Änderungen in Bruchteilen einer Sekunde.\\
Dennoch gibt es einige Punkte, die aufgrund der Mangelnden Zeit nicht getestet oder implementiert
werden konnten.\\
So ist beispielsweise das Verhalten des ABS abhängig von der Dauer der Hauptschleife. Normalerweise
ist dies kein Problem, wenn allerdings Debug-Ausgaben vorgenommen werden, kann das ABS sich
auf eine Art und Weise verhalten, die nicht erwünscht ist (dauerndes hin- und herschwenken der Räder,
da kein Schleifendurchlauf während des Nullpunktes stattfindet). Um dieses zu verhindern kann man
einen Timer-Interrupt mit einer möglichst niedrigen Auflösung benutzen. Der Vorteil wäre hierbei,
dass das ABS nicht mehr abhängig von der Geschwindigkeit der Hauptschleife ist, sowie, dass das
Abschalten des ABS gleichzusetzen ist mit dem Abschalten des zugehörigen Interrupts, was wiederum
den Overhead des ABS komplett eliminiert, wenn dieses abgeschaltet ist. Dieser Overhead wird auf
ungefähr 2,5 us pro Schleifendurchlauf geschätzt. Der Nachteil besteht in einem geringen
Mehraufwand, da ein Funktionsaufruf durch eine Interrupt-Service-Routine ersetzt wird.\\
Eine weitere Erweiterungsmöglichkeit für das System besteht in erhöhter Robustheit und erweiterter
Fehlererkennung. So ist es möglich eintreffende Befehle auf komplette syntaktische Korrektheit
zu überprüfen und nur solche Befehle zu akzeptieren, die diese Tests bestehen. Diese Erweiterung
muss allerdings möglichst effizient und einfach erweiterbar implementiert werden, um zum einen nicht
entgegen der Designprinzipien des Systems zu handeln, als auch das Laufzeitverhalten nicht
schwerwiegend zu beeinträchtigen. Zusätzlich zu dieser Fehlererkennen kann die Robustheit des Systems
durch ein Überwachungssystem erhöht werden, welches in periodischen Abständen, ermöglicht durch
die eingebauten Timer, die einzelnen Module des Gesamtsystems auf Anzeichen von Problemen untersucht,
wie z.B. voll Buffer, runaway Befehle, Parser Status Korruptionen und dergleichen. Um die Möglichkeit
von runaway Code auszuschließen, ist die Benutzung des eingebauten Watchdog-Timers möglich, dessen
timeout Wert allerdings sehr sorgfältig gewählt werden muss. Der timeout Wert darf nicht kleiner sein
als die längste Hauptschleifen Iteration plus ein entsprechendes Sicherheitspolster.\\
Ein Bereich, der während der Arbeit vollkommen vernachlässigt wurde, ist die Möglichkeit die
Motorplatine in einen Idle-Modus zu schicken, in diesem Modus verbraucht wie Platine wesentlich weniger
Strom. Dies würde die Zeit verlängern, die eine Batterie an solch einem Fahrzeug hält, während die
Praktikanten an ihm arbeiten und die Motorplatine längere Zeit nichts zu tun hat. In dieser Hinsicht
wäre es auch interessant den Idle-Modus bzw. das Verhalten um dem Idle-Modus durch Befehle während der
Laufzeit steuern zu können.\\
Wie im Kapitel über die Laufzeituntersuchung beschrieben ist die Datenrate über den I2C-Bus der größte
limitierende Faktor bei der Übermittlung von Fahrbefehlen von der Praktikumsplatine an die
Motorplatine. Damit dieses Problem minimiert wird kann man zum einen erwägen den I2C-Bus im \"high-speed\"
Modus operieren zu lassen, oder eine eigene Punkt-zu-Punkt Kommunikation mithilfe der freien Ports auf der
Motorplatine zu realisieren. Solch ein Maßgeschneideter Port mit einem eigens dafür entwickelten Protokoll
könnte die Latenz, die durch das Senden des Befehls entsteht, erheblich verringern.
