\documentclass[11pt,twoside]{scrbook}

\usepackage{a4}
\usepackage{german}
\usepackage{times}
\usepackage{graphicx}
\usepackage{verbatimfiles}
\usepackage{textcomp}
\usepackage[utf8]{inputenc}

\usepackage[outline,light]{draftcopy}

\begin{document}

%Deckblatt erzeugen
\thispagestyle{headings}
\title{Betriebssoftware für eine Fahrplattform unter besonderer Berücksichtigung der Echtzeitbedingungen}
\author{Christoph Peltz}

\maketitle
\setcounter{page}{1}
\pagenumbering{roman}

\thispagestyle{headings}
\begin{titlepage}
	\let\footnotesize\small
	\let\footnoterule\relax
	\null
	\vfil
	\vskip 60pt
	\begin{center}
		{\LARGE
			{\Large Technische Universit"at Braunschweig}\\
			Institut f"ur Betriebssysteme und
				Rechnerverbund\\[2cm]
			{\Large Bachelorarbeit}\\ [2cm]
			Betriebssoftware für eine Fahrplattform unter besonderer Berücksichtigung der Echtzeitbedingungen
			\par}%
		\vskip 6em
		{\large \lineskip .75em
		\begin{tabular}[t]{c}
			{\Large von}\\[.5em]
			{\Large cand. inform.\ Christoph Peltz}\\[7em]
			{\bf Aufgabenstellung und Betreuung:}\\[.5em]
			Dieter Brökelmann und
			Prof.\ Dr.-Ing.\ L.\ Wolf.\\
		\end{tabular}
		\par}%
		\vfill 
		{\large
			Braunschweig, den \today
			\par}%
	\end{center}
	\par
	% thanks
	\vfil
	\null
\end{titlepage}

~\newpage

% Erklaerung
\vspace*{7cm}
\centerline{\bf Erkl"arung}

\vspace*{1cm}
Ich versichere, die vorliegende Arbeit selbstst"andig und nur unter Benutzung
der angegebenen Hilfsmittel angefertigt zu haben.

\vspace*{3cm}
%\centerline{Braunschweig, den \today \hfil \hfil Unterschrift}
Braunschweig, den \today 

\pagestyle{headings}
\cleardoublepage

% Kurzfassung und Abstract

\centerline{\bf Kurzfassung}

Diese Bachelorarbeit umfasst die Entwicklung, Implementation als auch Untersuchung einer
Betriebssoftware für eine vorgegebene Plattform mit Blick auf den späteren Verwendungszweck
im Praktikum ''Programmierung eingebetteter Systeme''. Ziel ist es die aktuell verwendete
Betriebssoftware durch eine wartbarere, performante und erweiterbare Alternative zu
ersetzen. Dazu wird auch ein neues Protokoll entworfen, um die Befehlsverarbeitung bzw.
die Zusammensetzung dieser Befehle, auf beiden Kommunikationsseiten einfach und effizient
zu gestalten.

%
\vskip 3cm
%

\centerline{\bf Abstract}

This example abstract is indeed very abstract.

\cleardoublepage

\vspace*{7cm}
\centerline{[Hier wird sp"ater die Aufgabenstellung eingef"ugt.]}




\tableofcontents		% Inhaltsverzeichnis erzeugen
\cleardoublepage
\listoffigures			% Bilderverzeichnis erzeugen
\cleardoublepage
\listoftables			% Tabellenverzeichnis erzeugen
\cleardoublepage


%\parindent=0pt                   % erzeugt bei einem Absatz eine
%\parskip=6pt plus 3pt            % Leerzeile (kein Einruecken)

\setcounter{page}{0}

\pagestyle{headings}
\pagenumbering{arabic}

\chapter{Einleitung}

Im Praktikum ''Programmierung eingebetteter Systeme'' wird eine Motorplatine \cite{STUD_TIMO}
eingesetzt, damit die Studenten die Motoren benutzen können um ihr Fahrzeug
fahren zu lassen. Diese Motorplatine benötigt für seine Operation eine
Betriebssoftware, die die Befehle der Praktikumsplatine entgegen nimmt,
interpretiert und in Aktionen umsetzen kann. Die Entwicklung dieser Software,
sowie die qualitative Untersuchung ihres Laufzeitverhaltens ist Gegenstand
dieser Arbeit.\\
Zuerst wird auf die Hardware eingegangen für die die Betriebssoftware geschrieben wird
und die in einer eigenen Arbeit für diesen Einsatzzweck speziell entwickelt wurde.
Der Microcontroller und die Kommunikationseinrichtungen werden genauer betrachtet, um
einen Überblick über die Möglichkeiten zu haben, die die Motorplatine bietet.\\
Dann wird die Motorplatine mit der Hardware, die im Praktikum eingesetzt wird, in Verbindung
gebracht. Das beinhaltet die Beziehung zwischen der Praktikumsplatine, dem WLAN-Modul und der
Motorplatine. Dies ist die Platform, die im Praktikum benutzt wird und somit die Umgebung in
der die Software auch eingesetzt wird und deren Anforderungen sie genügen muss.\\
Für die Implementierung dieser Betriebssoftware wurde als Programmiersprache
C \cite{C_PROG} vorgegeben, deren Standard C99 ausgewählt wurde. Da diese
Software für längere Zeit eingesetzt werden soll, waren die Aspekte der
Wartbarkeit und die Möglichkeit die Software einfach erweitern zu können,
dominante Designaspekte, kurz darauf folgt die Anforderung, dass die Software
ihre Arbeit performant und zuverlässig ausführt.\\
Um die Performanz und Zuverlässigkeit zu Untersuchen wurde sowohl ein
Logikanalyzer als auch ein Oszilloskop benutzt, die Flanken und deren Länge
an nach außen gelegten Pins gemessen haben. Diese Pins wurden für den Zweck
der Untersuchung von der Betriebssoftware an wichtigen bzw. kritischen Stellen
im Programmcode gesetzt und wieder gelöscht.
		% Einleitung

\cleardoublepage

\chapter{Überblick über die Hardware}
Die Hardware, die in dieser Arbeit zum Einsatz kommt, wurde eigens für den Zweck des Praktikums
''Programmierung eingebetteter Systeme'' entwickelt.

\section{Einsatz im Praktikum}
Im vorgenannten Praktikum wird den Studenten der
Umgang mit kleinen Systemen näher gebracht, die sich wesentlich von den bekannten IBM-PC-kompatiblen
Systemen unterscheiden. Das schließt die Rechenleistung, den Speicherplatz und die
Kommunikationsmöglichkeiten mit ein. 
Der Großteil der Praktikanten kennt nur das Schreiben von Programmen, die auf Betriebssystemen laufen.
Dies ist mit gewissen Vorteilen verbunden. Es gibt
Dateien, der Speicher wird verwaltet, und vieles mehr. Auf diesen kleinen Systemen, mit denen die Studenten im
Praktikum konfrontiert werden, läuft kein Betriebssystem, sondern lediglich ein Bootloader. Dieser
ermöglicht es, neue Programme auf die Platine zu laden, um diese dann ausführen zu können.
Im Laufe des Praktikums lernen die Studenten, wie man Programme in der Sprache C schreibt, und wie die Timer
sowie andere angeschlossene Geräte verwendet werden, zu denen z.B. ein LCD und ein Servomotor zählen.
Außerdem wird den Studenten vermittelt, wie der I2C-Bus funktioniert und wie man diesen benutzt.\\
Im zweiten Teil des Praktikums können die Studenten sich für ein Projekt entscheiden,
welches sie durchführen möchten (die Projektideen kommen von den Studenten).
Als Plattform für ihre Projekte wird ihnen ein Fahrzeug zur Verfügung gestellt. Dieses
Fahrzeug besitzt die aus dem ersten Teil des Praktikums bekannte Praktikumsplatine und ein WLAN-Modul,
das das Programmieren der Praktikumsplatine
ohne Kabel ermöglicht. Außerdem können hierüber Daten mit der Praktikumsplatine während
der Laufzeit ausgetauscht werden. Zusätzlich besitzt das Fahrzeug eine Motorplatine mit
zugehörigen Motoren und Rädern, die zur Fortbewegung des Fahrzeugs genutzt werden. Die
Motorplatine wird mithilfe von Befehlen, welche die Praktikumsplatine sendet, gesteuert, und
diese wiederum steuert die Motoren.\\
Die Motorplatine, für die die Betriebssoftware in der vorliegenden Arbeit entwickelt wurde, wird
im Allgemeinen von den Studenten nur benutzt, aber nicht modifiziert (auch wenn ihnen dies
freisteht). Die Leistungen, die die Motorplatine bereitstellt, 
sollen einfach anzusprechen sein, damit die Studenten sich auf das Implementieren
ihres eigenen Projektes konzentrieren können.\\
%Bild des Fahrzeugs

\section{Die Motorplatine}
Die Motorplatine wurde im Zuge der Studienarbeit von Timo Klingeberg \cite{STUD_TIMO}
entwickelt.
\begin{figure}[htb]
 \centering
 \scalebox{0.5}{\includegraphics{pictures/board.png}}
 \caption{\label{board}Die Motorplatine}
\end{figure}
\subsection{Mikrocontroller}
Das Herzstück der Platine bildet ein Mikrocontroller der Firma Atmel.
Es handelt sich hierbei um einen ATMEGA2561\cite{ATMEGA_MANUAL}, der 256-KiB-Speicher für
Programme (Flash) hat, sowie 8-KiB-Speicher für Variablen (SRAM). Der maximale Takt für
diesen Mikrocontroller liegt bei 16 MHz, der auch ausgenutzt wird. Das bedeutet, dass
ein Takt 62,5 ns benötigt. Da der Großteil der Instruktionen des Mikrocontrollers nur
einen Takt benötigt, kann dieser theoretisch 16 MIPS leisten. Durch die Benutzung
von Funktionen, Pin-IO und bedingten Sprüngen bleibt dies aber nur eine theoretische Zahl.
Vorteilhafterweise unterstützt der Controller das ISP (In-System-Programming). Das war bei der
Entwicklung der Betriebssoftware von großem Nutzen, da hierdurch schnell und relativ
unkompliziert neue Versionen auf die Platine überspielt werden konnten.\\
Der Controller verfügt über 6 Timer. Zwei von diesen haben nur 8 Bit, die anderen vier
16 Bit. Das entspricht bei dem gegebenen Takt von 16 MHz einem maximalen Timer-Intervall von 262 ms.
Zusätzlich kann der Controller sechs Pulsweitenmodulationen betreiben, von
denen zwei für die Servomotoren genutzt werden, die die Räder antreiben.
\subsection{Ein-/Ausgabemöglichkeiten zur Praktikumsplatine}
Die Motorplatine verfügt sowohl über einen UART-Port, als auch einen I2C-Bus \cite{I2C_WIKI}, über die
die Praktikumsplatine mit der Motorplatine kommunizieren kann. Der UART-Port kann außerdem
für die Ausgabe von Debug-Informationen benutzt werden, wenn diese aktiviert wird. Im Allgemeinen
wird allerdings nur der I2C-Bus zur Kommunikation zwischen den beiden Platinen verwendet.
Da der Mikrocontroller über eingebaute Hardwarelogiken für den I2C-Bus und auch
den UART-Port verfügt, hält sich der administrative Aufwand für die Kommunikation in engen
Grenzen. Es können lediglich Interrupt-Service-Routinen (ISR) für diese Kommunikations-Einrichtungen zur Verfügung gestellt
werden. Dadurch ist es möglich, schnell auf Situationen zu reagieren, ohne Informationen verlieren zu können.
\subsection{Ein-/Ausgabe-Ports}
Zusätzlich zu den externen Kommunikationsmöglichkeiten mit der Praktikumsplatine besitzt der ATMEGA2561 54
programmierbare IO-Kanäle, die in Ports mit je 8 Kanälen zusammengefasst werden. 16 von diesen
Kanälen sind für die Steuerung der Servomotoren zuständig, die die Räder und die Hall-Sensoren betreiben.
Je ein Servomotor in Verbindung mit zwei Sensoren belegt einen Port. Der Servomotor benötigt 3 Kanäle
des Ports, und die Sensoren benötigen 5 Kanäle.
Diese Sensoren, die an den Motoren befestigt sind, lösen bei Bewegung der Räder Unterbrechungen aus.
Damit liefern sie Informationen, aus denen auf die Drehrichtung der Räder und die Strecke, die zurückgelegt wurde,
geschlossen werden kann.
Für jede Umdrehung eines Rades werden 360 Interrupts ausgelöst, d.h. für jedes Grad ein Interrupt. Je
nach Größe des Raddurchmessers ist eine Steckenauflösung von wenigen Millimetern möglich.
Durch die Rückmeldung der Sensoren ist es möglich, die Servomotoren nicht nur zu steuern, sondern zu regeln.
Zusätzlich kann die Geschwindigkeit der einzelnen Rädern aus diesen Informationen ermittelt werden.
Die Informationen werden genutzt, um die Pulsweiten-Modulation zu regeln, mit der die Servomotoren angetrieben
werden.\\
Neben diesen zwei Ports zum Regeln der Motoren sind weitere zwei Ports nach außen gelegt. Mithilfe
der Pins dieser zwei nicht belegten Ports konnte eine möglichst unverfälschte Untersuchung des
Echtzeitverhaltens der Betriebssoftware durchgeführt werden.

\subsection{LCD}
Die Motorplatine verfügt über einen Anschluss für ein LCD. Dieses LCD kann zur autonomen Ausgabe
von Informationen genutzt werden. Die Betriebssoftware nutzt diese Möglichkeit,
wenn ein LCD angeschlossen ist, und der zugehörige DIP-Schalter aktiviert wurde.
Es werden die aktuelle Version der Betriebssoftware, die Optionen und, falls vorhanden,
der aktuelle Befehl in hexadezimaler Darstellung ausgegeben. Zu den Optionen zählen der Status der Debug-Ausgaben,
welche Kommunikations-Einrichtung verwendet wird, und der Zustand des ABS. Diese Optionen
werden groß geschrieben, wenn sie aktiviert sind, und klein geschrieben, wenn sie
deaktiviert sind. Die Betriebssoftware ist auf LCDs eingestellt, die 4 Zeilen und 20 Zeichen je Zeile besitzen.
Sollte ein anderes LCD angeschlossen werden, ist eine korrekte Ausgabe nicht möglich.
	        % 2. Kapitel
\cleardoublepage

\chapter{Einsatz im Praktikum}
Im Praktikum ''Programmierung eingebetteter Systeme'' wird den Studenten der
Umgang mit kleinen Systemen näher gebracht, die sich wesentlich von den bekannten IBM-PC-kompatiblen
Systemen unterscheiden. Dies schließt die Rechenleistung, den Speicherplatz und die
Kommunikationsmöglichkeiten mit ein. Die meisten Praktikanten kennen nur das Programmieren
für ein Betriebssystem, bzw. immer mit dem Hintergedanken, dass das Programm auf einem
solchen Betriebssystem läuft. Dies ist mit gewissen Angenehmlichkeiten verbunden. Es gibt
Dateien, der Speicher wird verwaltet und vieles mehr. Auf diesen kleinen Systemen, mit denen die Studenten im
Praktikum konfrontiert werden, läuft kein Betriebssystem, lediglich ein Bootloader, der
es ermöglicht neue Programme auf die Platine zu laden, um diese dann ausführen zu können.
Im Laufe des Praktikums lernen die Studenten, wie man Programme in der Sprache C schreibt, wie die Timer
und andere angeschlossene Geräte verwendet werden, zu denen z.B. ein LCD und ein Servomotor zählen.
Außerdem wird den Studenten vermittelt, wie der I2C-Bus funktioniert und wie man diesen benutzt.\\
Im zweiten Teil des Praktikums können die Studenten sich für ein Projekt entscheiden,
welches sie durchführen möchten (diese Projektideen kommen von den Studenten).
Als Plattform für ihre Projekte wird ihnen ein Fahrzeug zur Verfügung gestellt. Dieses
Fahrzeug besitzt die aus dem ersten Teil des Praktikums bekannte Platine (von nun an
Praktikumsplatine genannt), ein WLAN-Modul, das das Programmieren der Praktikumsplatine
ohne Kabel ermöglicht, außerdem können hierüber Daten mit der Praktikumsplatine während
der Laufzeit ausgetauscht werden. Zusätzlich besitzt das Fahrzeug eine Motorplatine mit
zugehörigen Motoren und Rädern, die zur Fortbewegung des Fahrzeugs genutzt werden. Diese
Motorplatine wird mithilfe von Befehlen, die die Praktikumsplatine sendet, gesteuert und
diese wiederum steuert die Motoren.\\
Die Motorplatine, für die die Betriebssoftware in dieser Arbeit entwickelt wurde, wird
im Allgemeinen von den Studenten nur benutzt, nicht modifiziert (auch wenn ihnen dies
frei steht). Diese Leistungen, die die Motorplatine bereitstellt, sollten nicht die Studenten
behindern, sondern auch einfach anzusprechen sein, damit diese sich auf das Implementieren
ihres eigenes Projektes konzentrieren können.\\
%Bild des Fahrzeugs
		% 3. Kapitel
\cleardoublepage

\chapter{Design und Designentscheidungen}
Die Betriebssoftware wurde in Module aufgeteilt, wobei eine Code-Datei einem
Modul zugeordnet ist und ein Module mehrere Code-Dateien umfassen kann.
Es wurde besonderer Wert darauf gelegt, dass Module so wenig wie möglich andere
Module aufrufen und dies auch nur durch Funktionen und nicht durch Variablen.
Dieses Prinzip musste allerdings während der Optimierungs- und Testphase
geringfügig aufgeweicht werden, da entweder der Code dadurch unleserlich wurde
oder der Preis für einen Funktionsaufruf zu hoch war in Relation zu dem
Nutzen, der dadurch erzielt wurde.
Somit gibt es nun vier bis fünf globale Variablen, die sich während der Laufzeit
ändern können und in unterschiedlichen Modulen direkt referenziert werden.
Geändert werden diese allerdings nur in sehr wenigen Funktionen, in denen dies
auch explizit Kommentiert wurde. Außerdem wurde in den Coding-Guidelines extra
auf den Umstand hingewiesen wenn möglich keine Funktionen zu schreiben, die die
globalen Variablen verändern und wenn doch dies explizit zu dokumentieren.
Neben diesen maximal fünf Verhaltensverändernden globalen Variablen gibt es vier
weitere Variablen in zwei verschiedenen Modulen, die jeweils aus einem anderen
Modul explizit gesetzt werden. Dies sind die Trigger-Werte für die Position und
die Zeit. Diese werden nur durch den Drive, bzw den Advanced Drive Befehl
gesetzt und während der Motor Interrupts nach und nach dekrementiert.
		% 3. Kapitel
\cleardoublepage
                                % usw.
\chapter{Implementierung der Betriebssoftware}
Die Betriebssoftware wurde, wie in der Aufgabenstellung festgelegt, in C geschrieben.
Hierbei fiel die Wahl auf den Standard C99, der gegenüber C89 einige sprachliche
Aktualisierungen aufweist. Als Compiler wurde eine Version der
GNU-Compiler-Collection (GCC) mit einem Backend für AVR-kompatiblen Assembler benutzt.
Außer der C-Standard-Library und der AVR-IO-Library besitzt die Software keine externen
Abhängigkeiten im Code.

\section{Die Hauptschleife}
Die Hauptschleife ist in dieser Implementierung eine Endlosschleife, da die Beendigung dieser
Schleife sonst dazu führen würde, dass das System nicht mehr reagiert.
Wie in Abb. \ref{main_loop} zu sehen ist, werden in der Hauptschleife vier wichtige Funktionen
aufgerufen.

\subsection{process\_orders()}
Die process\_\-orders()\--Funktion bearbeitet die Befehle, die bereits in der Queue sind. Dafür holt
sich die Funktion den aktuellen Befehl von der Queue. Wenn es solch einen Befehl gibt, ruft die
Funktion eine Verteilerfunktion auf. Diese wiederum ruft die zugehörige Befehlsfunktion auf, indem
der Befehlscode als Index für eine Call-Table benutzt wird (siehe
Abb. \ref{dispatch_function}).\\
\begin{figure}[htb]
 \centering
 \scalebox{0.6}{\includegraphics{pictures/dispatch_function.png}}
 \caption{\label{dispatch_function}Die Verteilerfunktion}
\end{figure}
Falls kein Befehl vorliegt, oder die Queue angehalten wurde, ruft die process\_\-orders()-Funktion die
Funktion zum aktiven Bremsen auf. Das aktive Bremsen wird in Kapitel \ref{chapter_abs} behandelt.

\subsection{lcd\_update\_screen()}
Wenn ein LCD angeschlossen ist, können dort Informationen ausgegeben werden.
Der Anschluss des LCD wird dem System mit einem DIP-Schalter auf der Platine mitgeteilt.\\
Da das synchrone Aktualisieren des LCD sehr viel Zeit benötigt (vgl. Kapitel \ref{chapter_lcd_problem}), wird
während dieser Funktion maximal ein Zeichen an das LCD geschickt.
Dies wird durch Abfrage des Busy-Flags erreicht, welches signalisiert, dass das LCD noch
beschäftigt ist. Falls es nicht gesetzt ist und noch Daten zu aktualisieren sind,
wird das nächste Zeichen an das LCD gesendet.\\
Auf dem LCD werden die Versionsnummer der Betriebssoftware, der Status einiger globaler Variablen, und
der aktuell ausgeführte Befehl angezeigt. Wenn nun ein Befehl bearbeitet wird, der länger als einen
Schleifendurchlauf benötigt (das sind z.B. alle Fahr-Befehle), ruft die lcd\_\-update\_\-screen()-Funktion
die lcd\_\-update\_\-info()-Funktion auf, die diese Informationen in einem Puffer konstruiert. Nach und nach
gibt die lcd\_\-update\_\-screen()-Funktion den Inhalt dieses Puffers an das LCD weiter.\\
Befehle, die innerhalb eines Hauptschleifendurchlaufs abgearbeitet sind, werden nicht ausgegeben und
generieren auch keinen Aufruf von lcd\_\-update\_\-info(). Das ist nötig, weil diese Befehle zu schnell abgearbeitet
werden. Es könnten mitunter vier bis fünf dieser Befehle abgearbeitet werden, bevor das LCD auch nur einmal vollständig
aktualisiert werden kann.

\subsection{parser\_update()}
Der Parser ist dafür zuständig, aus den Bytes, die über I2C oder UART gelesen werden, Befehle in Form von
order\_t-Strukturen zu erstellen. Die parser\_\-update()-Funktion fragt beim IO-Modul nach, wie viele Bytes
zum Abholen bereit stehen. Diese werden dann geholt und an die Funktion parser\_\-add\_\-byte() übergeben.\\
Diese parser\_\-add\_\-byte()-Funktion fügt das Byte an die richtige Stelle im Puffer ein. Wenn ein Befehl
komplett ist, dies wird mit der parser\_\-order\_\-complete()-Funktion überprüft, gilt der Befehl als fertig und
alle weiteren Bytes, die hinzugefügt werden, landen in einer neuen order\_t-Struktur.\\
Zum Erkennen, wann ein Befehl zu Ende ist, benutzt die parser\_\-order\_\-complete()-Funktion
die bytes\_\-needed()-Funktion. In dieser ist fest codiert, welcher Befehlscode mit welchen Optionen wie viele
Bytes benötigt. Das ist auch eine der Stellen, die angepasst werden müssen, wenn neue Befehle hinzugefügt
oder bestehende verändert werden sollen.

\subsection{queue\_update()\label{chapter_queue_update}}
Diese Funktion führt Wartungsarbeiten an der Befehlswarteschlange (Queue) durch. Das beinhaltet, neue
Befehle beim Parser-Modul abzuholen und diese korrekt einzureihen. Es gibt zwei Möglichkeiten, wie die Queue
diese neuen Befehle einreihen kann. Zum einen als normale Befehle, die einer nach dem anderen abgearbeitet werden,
zum anderen als priorisierter Befehl. Es kann nur ein priorisierter Befehl in der Queue sein. Die Befehle werden
umgehend in der nächsten Haupt\-schleifen\-iteration ausgeführt. In die Kategorie der priorisierten Befehle fallen
alle Queue-Kontroll-Befehle, wie z.B. pausieren, löschen, aktuellen Befehl verwerfen etc. (vgl. Kapitel \ref{chapter_protokoll}).

\section{Das Aktive-Brems-System (ABS)\label{chapter_abs}}
Das aktive Bremssystem soll bewirken, dass die Räder im praktischen Betrieb ihre Position verlassen können.
Dies wird realisiert, indem
eine Referenz-Position für jedes Rad gespeichert wird. Während der Hauptschleife wird die tatsächliche Position mit
der Referenz-Position verglichen. Für den Fall, dass diese Positionen nicht übereinstimmen, werden die Motoren mit einer
einstellbaren Geschwindigkeit so betrieben, dass die Räder wieder auf die Referenz-Position gebracht werden.\\
Die Referenz-Positionen werden an drei verschiedenen Stellen im Code gesetzt. Zum einen in der process\_\-orders()-Funktion
in der Hauptschleife, wenn der aktuelle Befehl beendet wurde, zum anderen in den Fahr-Befehls-Funktionen, falls ein Rad
früher als das andere seine Stopp-Bedingung erreicht hat.\\
Das ABS kann vom Benutzer bei laufendem System angepasst werden. So kann man die Geschwindigkeit ändern, mit der
die Motoren die Positions-Differenz ausgleichen. Außerdem kann man Teile des ABS deaktivieren oder auch reaktivieren,
oder das gesamte ABS abschalten bzw. wieder anschalten. So kann der Benutzer das ABS an seine Wünsche anpassen.

\section{Befehle: Struktur und Funktionen}
Die Struktur (Abb. \ref{order_type}), die einen Befehl im System repräsentiert, besteht hauptsächlich aus einem Array, in dem die eigentlichen Daten
gespeichert sind, und einem Status-Byte, in dem Statusinformationen in Form von Flags gespeichert werden.
\begin{figure}[htb]
 \centering
 \scalebox{0.6}{\includegraphics{pictures/order_t.png}}
 \caption{\label{order_type}Befehlsstruktur}
\end{figure}
Das erste Byte dieses Arrays ist das Kommando-Byte, welches die Art des Befehls und die zugehörigen Optionen spezifiziert. Der
Befehlscode 0x00 ist für zukünftige Erweiterungen reserviert, die mehr als ein Kommando-Byte
benötigen. Des Weiteren sind die Befehlscodes 0x01 bis 0x06 durch diese Arbeit bereits definiert und mit Funktionalität erfüllt.
Die Befehlscodes 0x07 bis 0x0f sind noch nicht definiert und können für zukünftige Erweiterungen benutzt werden, die mit \textbf{einem}
Kommando-Byte auskommen.\\
Alle auf das Kommando-Byte (oder die Kommando-Bytes im Falle des Befehlscodes 0x00) folgenden Bytes sind Parameter. Deren Anzahl und Länge
hängt von der Spezifikation des Befehls ab.
Bei Parametern, die länger als ein Byte sind, wird zuerst das höchstwertige Byte im Array gespeichert. 
Dann folgen die übrigen Bytes mit absteigender Wertigkeit.\\
Oft benutzte Aktionen bezüglich der Befehlsstruktur wurden zusammengefasst (siehe Abb. \ref{order_init} und \ref{order_copy}).
Vor der Untersuchung des Laufzeitverhaltens, und der damit einhergehenden Optimierungen, wurden diese beiden Funktionen
durch for-Schleifen implementiert. Wie sich bei der Untersuchung herausstellte, waren die for-Schleifen aber langsamer
als die Standard-C-Funktionen memset() und memcpy().
\begin{figure}[htb]
 \centering
 \scalebox{0.6}{\includegraphics{pictures/order_init.png}}
 \caption{\label{order_init}order\_init()-Funktion}
\end{figure}
\begin{figure}[htb]
 \centering
 \scalebox{0.6}{\includegraphics{pictures/order_copy.png}}
 \caption{\label{order_copy}order\_copy()-Funktion}
\end{figure}

\section{Datenpfad von Befehlen}
Befehle werden entweder über die UART- oder die I2C-Schnittstelle byteweise empfangen. Diese Schnittstellen werden über
Interrupt-Service-Routinen bearbeitet, um zeitnah auf eingehende Daten zu reagieren, da diese Übertragung die größte
Latenz-Quelle darstellt (I2C: ca. 270 \textmu{}s pro Byte; UART: ca 139 \textmu{}s pro Byte). In diesen Interrupt-Service-Routinen wird
das empfangene Byte in den Eingangspuffer gelegt. Wenn die Hauptschleife wieder die parser\_\-update()-Funktion erreicht,
werden die bisher empfangenen Bytes abgeholt und in dem Parser-Puffer abgelegt, um aus den Bytes order\_t-Struktur-Instanzen
zu generieren. Wenn der Befehl fertig im Parser vorliegt, ruft ihn queue\_\-update() ab und reiht ihn in die Warteschlange
ein (siehe Kapitel \ref{chapter_queue_update}). Von der Warteschlange holt sich die process\_\-orders()-Funktion den
aktuellen Befehl und führt die zugeordnete order\_function-Funktion solange aus, bis im Status-Byte (vgl. Abb. \ref{order_type})
das ORDER\_\-STATUS\_\-DONE-Flag gesetzt wurde. Anschließend entfernt sie diesen aus der Warteschlange.\\
Falls der zu bearbeitende Befehl eine Ausgabe von Daten bewirkt, werden diese in der entsprechenden order\_\-function-Funktion
in dem Ausgabepuffer des IO-Frameworks angereiht. Dieses Framework wird dann, sobald wie möglich, diese Daten
über die ausgewählte Schnittstelle senden (bei UART wird sofort mit dem Senden begonnen; bei I2C muss gewartet werden, bis die Daten
vom Benutzer abgerufen werden).
% Datenpfadbild

\section{Debug-Ausgaben \label{impl_debug}}
Es ist nicht ohne weiteres möglich, einen üblichen Debugger zur Fehlerfindung zu verwenden; stattdessen
sind Debug-Ausgaben notwendig, um die Funktion des Debuggers zu ersetzen.
Im Gegensatz zu einem Debugger muss zusätzlicher Programmcode eingefügt werden, um Debug-Ausgaben
realisieren zu können.
Diese Ausgaben beeinträchtigen die Geschwindigkeit des gesamten Systems auch dann noch, wenn
diese mit if-Statements umschlossen werden (siehe Abb. \ref{debug_trick}).

Während der normalen Operation der Betriebssoftware im Praktikum sind diese Ausgaben
nicht nötig, und für die meisten Teilnehmer des Praktikums nicht informativ.
Der Leser muss eine entsprechende Kenntnis des Codes vorweisen, damit die Ausgaben
zum Zweck der Fehlerbehebung benutzt werden können.\\
Wegen der geringen Hilfe, die diese Ausgaben im Normalfall dem Praktikanten bieten,
und der negativen Auswirkung der Ausgaben auf die Performance des Systems, wurde
ein Prä-Prozessor-Technik angewandt, die zusammen mit der eingestellten Stufe der
Code-Optimierung des Compilers die beiden Probleme löst.

Durch diese Technik werden die Debug-Ausgaben nur dann kompiliert,
wenn dem Compiler der Parameter -DDEBUG übergeben wird. 
Zwar ist es auch dann noch möglich, mit einem DIP-Schalter auf der Platine die Ausgaben
auszuschalten, aber es bleiben immer noch die leicht negativen Auswirkungen auf die Performance.

Normalerweise
wird die Betriebssoftware ohne diesen Parameter kompiliert. Das bedeutet, dass die Software neu kompiliert werden muss, wenn
diese Debug-Ausgaben erwünscht sind.
\begin{figure}[htb]
 \centering
 \scalebox{0.5}{\includegraphics{pictures/debug_trick.png}}
 \caption{\label{debug_trick}Debug-Defines und die Verwendung im Code}
\end{figure}

\section{IO-Framework}
Das Ziel des IO-Frameworks war die Abstraktion der Ein- und Ausgabe von Daten der zu Grunde liegenden Schnittstellen.
Die UART- und die I2C-Schnittstelle sind in der Benutzung sehr unterschiedlich. Statt überall im Code, wo E/A stattfindet,
jeweils für beide Schnittstellen Code einzufügen, wurde das IO-Framework als Zwischenschicht entwickelt. Es verfügt sowohl
über einen Ausgabe- als auch einen Eingabepuffer. Diese sind jeweils auf 256 Bytes festgelegt. Durch diese Definition
konnte das normale Überlaufverhalten der 8-Bit-Variablen ausgenutzt werden, um Modulo-Operationen zu ersetzen, welche
unnötigerweise viel Zeit in Anspruch nehmen.\\
Eine Besonderheit ist die Ausgabe von Daten auf Objekt-Basis. Ein Objekt hat mindestens ein Byte und maximal 256 Byte. Objekte
werden entweder komplett übertragen oder gar nicht. Wenn ein Objekt nicht komplett übertragen werden konnte, wird bei der
nächsten Übertragung vom Anfang des Objektes wieder angefangen. Außerdem bewirkt die Ausgabe von Objekten bei Benutzung der
I2C-Schnittstelle, dass für jede Lese-Operation, die von der Praktikumsplatine eingeleitet wird, ein Objekt übermittelt wird.\\
E/A-Operationen geschehen nicht-blockend und gepuffert. Damit verbraucht die E/A nur Prozessorzeit, wenn es nötig ist, und
vermeidet so nutzlosen Zeitverbrauch bedingt durch aktives Warten.

\section{Unterstützende Bibliothek für die Praktikumsplatine}
Das Benutzen der Motorplatine soll für die Studierenden möglichst einfach sein.
Aus diesem Grund wurde neben der Betriebssoftware für die Motorplatine auch eine
Bibliothek für die Praktikumsplatine geschrieben. Diese Bibliothek stellt
Funktionen und Definitionen zur Verfügung, um Befehle an die Motorplatine zu
senden oder zu empfangen. Hierfür wurde eine modifizierte Version der
Befehls-Struktur für die Praktikumsplatine geschrieben. Es sind drei Funktionsaufrufe
nötig, um jeden möglichen Befehl zusammen zu stellen und zu versenden. Die erste
Funktion setzt das Kommando-Byte. Die zweite setzt die Parameter, dies wurde
mit Funktion gelöst, die eine variable Liste von Argumenten erhält. Die dritte und
letzte Funktion sendet den Befehl an die Motorplatine. Vorher muss allerdings die
Befehls-Struktur einmal initialisiert werden.
\begin{verbatim}
order_t order;
order_init(&order);
order_set_type(&order, ORDER_DRIVE_P_P);
order_add_params(&order, "1122", 127, -100, 32700, 16768);
order_send(&order);
\end{verbatim}
		% 3. Kapitel
\cleardoublepage

\chapter{Untersuchung des Laufzeitverhaltens}
Die Untersuchung der Laufzeitverhaltens wurde mithilfe eines Logik-
Analysers durchgeführt. Ein Logikanalyser misst wie lange und wann
ein Pin 1 und/oder 0 ist. Die Pins, die nötig sind um diese Messungen
durchzuführen, dürfen noch nicht belegt sein. Vorteilhafterweise
besitzt die Platine zwei Ports, die noch nicht belegt sind aber
trotzdem nach draußen gelegt wurden, somit konnte der Logik-
Analyser an die Platine angeschlossen und ein kleines Modul
geschrieben werden, um diese Pins setzen und wieder löschen zu
können. Diese Operationen wurden als Makros implementiert um den
Messungsoverhead so gering wie möglich zu halten.
\begin{center}
	\begin{tabular}{|c||c|c|}
		\hline
		\textbf{Makro} & \textbf{benötigte Takte} & \textbf{benötigte Zeit bei 16 MHz} \\ \hline \hline
		pin\_set() & 2 & 125 ns \\ \hline
		pin\_clear() & 2 & 125 ns \\ \hline
		pin\_toggle() & 4 & 250 ns \\ \hline
	\end{tabular}
\end{center}
Wie hier beschrieben sind die Zeiten für das Ausführen der Instruktion
zum setzen, löschen und umschalten von einzelnen Pins ziemlich gering.
Doch bei den Messungen insbesondere mit einem Leistungsfähigen und
sehr genauen Oszilloskop konnte herausgefunden werden, dass das eigentliche
wechsel des Stroms am Pin verhältnissmäßig langsam durchgeführt wird,
insbesondere das Abfallen des Stromes, also bei einer fallenden Flanke
benötigt ungefähr 2 us von denen allerdings, und hier liegt das Problem,
zwischen 0.5 und 1 us falschlicherweise als "high" gemessen wird.
D.h. der Logikanalyser misst eine gewisse Zeitspanne einen "falschen" Wert.
(Er ist nicht physikalisch falsch nur logisch). Denn wenn der Strom abfällt
ist der bin schon nicht mehr gesetzt, der Logikanalyser allerdings betrachtet
dies teilweise immernoch als gesetzt.
Aufgrund dieses Umstandes als auch der Tatsache, dass das System nicht untersucht
werden kann ohne einen geringen Fußabdruck zu hinterlassen, sind die Messungen mit
einem abschätzbaren aber nicht genau vorhersagbaren Fehler im Vergleich zur
Wirklichkeit behaftet.
		% 3. Kapitel
\cleardoublepage

\chapter{Zusammenfassung und Ausblick}
Das Ergebniss dieser Arbeit ist, dass die Platine auf Befehle schnell und korrekt reagiert,
solange diese nicht gravierend von den Spezifikation abweichen. Es gehen keine Interrupts verloren, und die
Platine reagiert auf Änderungen in Bruchteilen einer Sekunde.\\
Dennoch gibt es einige Punkte, die aufgrund der begrenzten Zeit nicht getestet oder implementiert
werden konnten.\\
So ist beispielsweise das Verhalten des ABS abhängig von der Dauer der Hauptschleife. Normalerweise
ist dies kein Problem. Wenn allerdings Debug-Ausgaben vorgenommen werden, kann das ABS sich
auf eine Art und Weise verhalten, die nicht erwünscht ist (dauerndes hin und her schwenken der Räder,
da kein Schleifendurchlauf während des Nullpunktes stattfindet). Um dies zu verhindern, kann man
einen Timer-Interrupt mit einer möglichst niedrigen Auflösung benutzen. Der Vorteil wäre hierbei,
dass das ABS nicht mehr abhängig von der Geschwindigkeit der Hauptschleife ist. Außerdem ist das
Abschalten des ABS gleichzusetzen mit dem Abschalten des zugehörigen Interrupts. Dieses wiederum
eliminiert den Overhead des ABS komplett, wenn es abgeschaltet ist. Dieser Overhead wird auf
ungefähr 2,5 us pro Schleifendurchlauf geschätzt. Der Nachteil besteht in einem geringen
Mehraufwand, da ein Funktionsaufruf durch eine Interrupt-Service-Routine ersetzt wird.\\
Eine weitere Erweiterungsmöglichkeit für das System besteht in erhöhter Robustheit und erweiterter
Fehlererkennung. So ist es möglich eintreffende Befehle auf komplette syntaktische Korrektheit
zu überprüfen und nur solche Befehle zu akzeptieren, die diese Tests bestehen. Diese Erweiterung
muss allerdings möglichst effizient und einfach erweiterbar implementiert werden, um zum einen nicht
entgegen der Designprinzipien des Systems zu handeln, als auch das Laufzeitverhalten nicht
schwerwiegend zu beeinträchtigen. Zusätzlich zu dieser Fehlererkennung kann die Robustheit des Systems
durch ein Überwachungssystem erhöht werden, welches in periodischen Abständen, ermöglicht durch
die eingebauten Timer, die einzelnen Module des Gesamtsystems auf Anzeichen von Problemen untersucht,
wie z.B. volle Puffer, runaway Befehle, Parser Status Korruption und dergleichen. Um die Möglichkeit
von runaway Code auszuschließen, ist die Benutzung des eingebauten Watchdog-Timers möglich, dessen
timeout Wert allerdings sehr sorgfältig gewählt werden muss. Der timeout Wert darf nicht kleiner sein
als die längste Hauptschleifen Iteration plus ein entsprechendes Sicherheitspolster.\\
Ein Bereich, der während der Arbeit vollkommen vernachlässigt wurde, ist die Möglichkeit die
Motorplatine in einen Idle-Modus zu schicken, in diesem Modus verbraucht wie Platine wesentlich weniger
Strom. Dies würde die Zeit verlängern, die eine Batterie an solch einem Fahrzeug hält, während die
Praktikanten an ihm arbeiten und die Motorplatine längere Zeit nichts zu tun hat. In dieser Hinsicht
wäre es auch interessant den Idle-Modus bzw. das Verhalten um dem Idle-Modus durch Befehle während der
Laufzeit steuern zu können.\\
Wie im Kapitel über die Laufzeituntersuchung beschrieben ist die Datenrate über den I2C-Bus der größte
limitierende Faktor bei der Übermittlung von Fahrbefehlen von der Praktikumsplatine an die
Motorplatine. Damit dieses Problem minimiert wird kann man zum einen erwägen den I2C-Bus im \"high-speed\"
Modus operieren zu lassen, oder eine eigene Punkt-zu-Punkt Kommunikation mithilfe der freien Ports auf der
Motorplatine zu realisieren. Solch ein Maßgeschneiderter Port mit einem eigens dafür entwickelten Protokoll
könnte die Latenz, die durch das Senden des Befehls entsteht, erheblich verringern.
		% Zusammenfassung und Ausblick
\cleardoublepage

\addcontentsline{toc}{chapter}{Literaturverzeichnis}
\bibliographystyle{unsrt}
\bibliography{thesis}
\cleardoublepage

\begin{appendix}
\begin{figure}[htb]
 \centering
 \scalebox{0.3}{\includegraphics{pictures/modules.png}}
 \caption{\label{modules}Die Module und ihre Beziehungen}
\end{figure}
                                % usw.
\cleardoublepage

\end{appendix}

\end{document}
