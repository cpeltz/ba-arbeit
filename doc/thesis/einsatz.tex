\chapter{Einsatz im Praktikum}
Im Praktikum ''Programmierung eingebetteter Systeme'' wird den Studenten der
Umgang mit kleinen Systemen, die sich wesentlich von den bekannten IBM-PC-kompatiblen
Systemen unterscheiden. Sowohl was die Rechenleisten, Speicherplatz, als auch
Kommunikationsmöglichkeiten angeht. Die meisten Praktkanten kennen nur das Programmieren
für ein Betriebssystem, bzw immer mit dem Hintergedanken, dass das Programm auf einem
solchen Betriebssystem läuft. Auf diesen kleinen Systemen, mit denen die Studenten im
Praktikum konfrontiert werden, läuft kein Betriebssystem, lediglich ein Bootloader, der
es ermöglicht neue Programme auf die Platine zu laden, um diese dann auch ausführen zu können.
Im Laufe des Praktikums lernen die Studenten, wie man Programme in C schreibt, wie die Timer
und andere angeschlossene Geräte verwendet werden, wie z.B. ein LCD oder ein Servomotor.
Außerdem wird den Studenten vermittelt, wie der I2C-Bus funktioniert und wie man diesen benutzt.\\
Im zweiten Teil des Praktikums können die Studenten sich für ein Projekt entscheiden,
welches sie durchführen möchten (diese Projektideen kommen auch von den Studenten).
Als Platform für ihre Projekte wird ihnen ein Fahrzeug zur Verfügung gestellt. Dieses
Fahrzeug besitzt die aus dem ersten Teil des Praktikums bekannte Platine (von nun an
Praktikumsplatine genannt), ein WLAN-Modul, das das Programmieren der Praktikumsplatine
ohne Kabel ermöglicht, außerdem können hierüber Daten mit der Praktikumsplatine während
der Laufzeit ausgetauscht werden. Zusätzlich besitzt das Fahrzeug eine Motorplatine mit
zugehörigen Motoren und Rädern, die zur Fortbewegung des Fahrzeugs benutzt werden. Diese
Motorplatine wird mithilfe von Befehlen, die die Praktikumsplatine sendet, gesteuert und
diese wiederum steuert die Motoren.\\
Die Motorplatine, für die die Betriebssoftware in dieser Arbeit entwickelt wurde, wird
im Allgemeinen von den Studenten nur benutzt, nicht modifiziert (auch wenn ihnen dies
freisteht). Diese Leistungen, die die Motorplatine bereitstellt, sollten nicht die Studenten
behindern, sondern auch einfach anzusprechen sein, damit diese sich auf das Implementieren
ihres eigenes Projektes konzentrieren können.\\
%Bild des Fahrzeugs
