\chapter{Einleitung}

Im Praktikum ''Programmierung eingebetteter Systeme'' wird eine Motorplatine \cite{STUD_TIMO}
eingesetzt, damit die Studenten die Motoren benutzen können um ihr Fahrzeug
fahren zu lassen. Diese Motorplatine benötigt für seine Operation eine
Betriebssoftware, die die Befehle der Praktikumsplatine entgegen nimmt,
interpretiert und in Aktionen umsetzen kann. Die Entwicklung dieser Software,
sowie die qualitative Untersuchung ihres Laufzeitverhaltens ist Gegenstand
dieser Arbeit.\\
Zuerst wird auf die Hardware eingegangen für die die Betriebssoftware geschrieben wird
und die in einer eigenen Arbeit für diesen Einsatzzweck speziell entwickelt wurde.
Der Microcontroller und die Kommunikationseinrichtungen werden genauer betrachtet, um
einen Überblick über die Möglichkeiten zu haben, die die Motorplatine bietet.\\
Dann wird die Motorplatine mit der Hardware, die im Praktikum eingesetzt wird, in Verbindung
gebracht. Das beinhaltet die Beziehung zwischen der Praktikumsplatine, dem WLAN-Modul und der
Motorplatine. Dies ist die Platform, die im Praktikum benutzt wird und somit die Umgebung in
der die Software auch eingesetzt wird und deren Anforderungen sie genügen muss.\\
Für die Implementierung dieser Betriebssoftware wurde als Programmiersprache
C \cite{C_PROG} vorgegeben, deren Standard C99 ausgewählt wurde. Da diese
Software für längere Zeit eingesetzt werden soll, waren die Aspekte der
Wartbarkeit und die Möglichkeit die Software einfach erweitern zu können,
dominante Designaspekte, kurz darauf folgt die Anforderung, dass die Software
ihre Arbeit performant und zuverlässig ausführt.\\
Um die Performanz und Zuverlässigkeit zu Untersuchen wurde sowohl ein
Logikanalyzer als auch ein Oszilloskop benutzt, die Flanken und deren Länge
an nach außen gelegten Pins gemessen haben. Diese Pins wurden für den Zweck
der Untersuchung von der Betriebssoftware an wichtigen bzw. kritischen Stellen
im Programmcode gesetzt und wieder gelöscht.
