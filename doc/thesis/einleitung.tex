\chapter{Einleitung}

Im Praktikum zur Programmierung eingebetteter Systeme wird eine Motorplatine \cite{STUD_TIMO}
eingesetzt, damit die Studenten die Motoren benutzen können um ihr Fahrzeug
fahren zu lassen. Auf der Motorplatine muss damit ein Betriebsprogram laufen,
dem die Studenten, die die Praktikumsplatine programmieren, Befehle schicken.
Diese Befehle bewirken unter anderem, dass die an die Motorplatine angeschlossenen
Motoren aktiviert werden und sich so das Fahrzeug auf Befehl der Praktikumsstudenten
fortbewegt. Das Betriebsprogramm der Motorplatine, welches im Laufe dieser Arbeit
entwickelt wurde, stellt die Schnittstelle zwischen der eigentlichen Motor-Hardware
und der Praktikumsplatine dar.
