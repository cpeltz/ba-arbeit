\chapter{Design und Designentscheidungen}
Die Betriebssoftware wurde in Module aufgeteilt, wobei eine Code-Datei einem
Modul zugeordnet ist und ein Module mehrere Code-Dateien umfassen kann.
Es wurde besonderer Wert darauf gelegt, dass Module so wenig wie möglich andere
Module aufrufen und dies auch nur durch Funktionen und nicht durch Variablen.
Dieses Prinzip musste allerdings während der Optimierungs- und Testphase
geringfügig aufgeweicht werden, da entweder der Code dadurch unleserlich wurde
oder der Preis für einen Funktionsaufruf zu hoch war in Relation zu dem
Nutzen, der dadurch erzielt wurde.
Somit gibt es nun vier bis fünf globale Variablen, die sich während der Laufzeit
ändern können und in unterschiedlichen Modulen direkt referenziert werden.
Geändert werden diese allerdings nur in sehr wenigen Funktionen, in denen dies
auch explizit Kommentiert wurde. Außerdem wurde in den Coding-Guidelines extra
auf den Umstand hingewiesen wenn möglich keine Funktionen zu schreiben, die die
globalen Variablen verändern und wenn doch dies explizit zu dokumentieren.
Neben diesen maximal fünf Verhaltensverändernden globalen Variablen gibt es vier
weitere Variablen in zwei verschiedenen Modulen, die jeweils aus einem anderen
Modul explizit gesetzt werden. Dies sind die Trigger-Werte für die Position und
die Zeit. Diese werden nur durch den Drive, bzw den Advanced Drive Befehl
gesetzt und während der Motor Interrupts nach und nach dekrementiert.
