\chapter{Untersuchung des Laufzeitverhaltens}
Die Untersuchung der Laufzeitverhaltens wurde mithilfe eines Logik-
Analysers durchgeführt. Ein Logikanalyser misst wie lange und wann
ein Pin 1 und/oder 0 ist. Die Pins, die nötig sind um diese Messungen
durchzuführen, dürfen noch nicht belegt sein. Vorteilhafterweise
besitzt die Platine zwei Ports, die noch nicht belegt sind aber
trotzdem nach draußen gelegt wurden, somit konnte der Logik-
Analyser an die Platine angeschlossen und ein kleines Modul
geschrieben werden, um diese Pins setzen und wieder löschen zu
können. Diese Operationen wurden als Makros implementiert um den
Messungsoverhead so gering wie möglich zu halten.
\begin{center}
\begin{tabular}{|c||c|c|}
\hline
\textbf{Makro} & \textbf{benötigte Takte} & \textbf{benötigte Zeit bei 16 MHz} \\ \hline \hline
pin\_set() & 2 & 125 ns \\ \hline
pin\_clear() & 2 & 125 ns \\ \hline
pin\_toggle() & 4 & 250 ns \\ \hline
\end{tabular}
\end{center}
