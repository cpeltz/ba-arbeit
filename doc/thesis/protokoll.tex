\chapter{Protokoll}
Die Praktikumsplatine muss, um das Fahrzeug in Bewegung zu setzen, mit
der Motorplatine kommunizieren. Diese Kommunikation muss sollte möglichst
effizient erfolgen. Deswegen ist die Entwicklung eines guten Protokolls
sehr wichtig für das gesamte System.\\
Das Verläufer-Protokoll benutzte Zeichenketten, um Instruktionen zu übermitteln.
Dies hat zwei wichtige Nachteile. Der erste ist, dass das Zusammensetzen der
Zeichenketten mithilfe der verfügbaren Prozessoren, sehr viel Rechenleistung in
Anspruch nimmt. Der zweite Nachteil ist, dass durch die Verwendung der Zeichenketten
die Informationsdichte der Instruktionen nicht sehr hoch ist und damit zusätzliche
Wartezeit bei der Übermittlung anfällt.

\section{Grundlegende Konzepte}
Da die Kommunikationseinrichtungen der Hardware Byte-orientiert sind, wurde das
Protokoll ebenfalls Byte-orientiert aufgebaut. Im Gegensatz zu dem Vorgänger-Protokoll
werden also keine Zeichenketten benutzt, um die Informationen zu repräsentieren, sondern
nur der Wert der einzelnen Bytes.\\
Ein Befehl ist eine Folge von n Bytes, wobei n mindestens 1 und maximal die im Code
eingestellte Größe der Befehlsstruktur entspricht. Von den eingebauten Befehlen besitzt
keiner eine maximale Länge von mehr als 9 Bytes. Das erste Byte eines Befehls hat eine
besondere Bedeutung. Dieses Byte wird Kommando-Byte genannt. Dieses Kommando-Byte
ist in zwei Teile geteilt. Der erste Teil umfasst die niederwertigsten 4 Bit und
wird Befehlscode genannt. Dieser Befehlscode spezifiziert die Art des Befehls.
Der zweite Teil beinhaltet die höchstwertigen 4 Bit. In ihm werden Optionen
angegeben, die den im Befehlscode angegebenen Befehl modifizieren. Die Kombination
von Befehlscode und Optionen legt auch die Länge des Befehls fest. Alle auf dem
Kommando-Byte folgende Bytes sind Parameter, wie z.B. die Geschwindigkeit der Räder.\\
Da 4 Bit für Befehlscodes zur Verfügung stehen, sind 16 verschiedene Befehle möglich.
Sechs Befehle wurden im Zuge dieser Arbeit implementiert, das würde noch Platz für
10 weitere Befehle lassen. Das Protokoll sollte allerdings noch mehr Freiheiten
für zukünftige Erweiterungen bieten, deswegen wurde einer der Befehlscodes
reserviert. Dieser reservierte Befehlscode und dessen Behandlung im Code, ermöglichen
es, dass es mehr als ein Kommando-Byte gibt. Damit ist es möglich Befehle einfach 
hinzuzufügen, die nicht auf das ''ein Kommando-Byte, viele Parameter''-Schema passen.\\
Wenn Parameter übertragen werden müssen, die mehr als ein Byte benötigen, wird zuerst
das höherwertigste Byte übertragen. Danach absteigend nach der Wertigkeit die anderen
Bytes.\\
Durch die volle Ausnutzung der Bytes gibt es kein Protokoll-spezifisches STOP- oder START-Byte.
Das bedeutet, dass das Protokoll sich auf die Flusskontrolle der zugrunde liegende Hardware
verlässt.

\section{Eingebaute Befehle}
Sechs Befehle und die Infrastruktur für den reservierten Befehlscode wurden implementiert.
Im nachfolgenden werden die Befehle einzeln vorgestellt.

\subsection{Extended Instruction}
\subsection{Control}
\subsection{Queue}
\subsection{Drive}
\subsection{Advanced Drive}
\subsection{SetPID}
\subsection{Option}
