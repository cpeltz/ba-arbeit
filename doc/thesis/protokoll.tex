\chapter{Protokoll\label{chapter_protokoll}}
Die Praktikumsplatine muss, um das Fahrzeug in Bewegung zu setzen, mit
der Motorplatine kommunizieren. Diese Kommunikation muss sollte möglichst
effizient erfolgen. Deswegen ist die Entwicklung eines guten Protokolls
sehr wichtig für das gesamte System.\\
Das Verläufer-Protokoll benutzte Zeichenketten, um Instruktionen zu übermitteln.
Dies hat zwei wichtige Nachteile. Der erste ist, dass das Zusammensetzen der
Zeichenketten mithilfe der verfügbaren Prozessoren, sehr viel Rechenleistung in
Anspruch nimmt. Der zweite Nachteil ist, dass durch die Verwendung der Zeichenketten
die Informationsdichte der Instruktionen nicht sehr hoch ist und damit zusätzliche
Wartezeit bei der Übermittlung anfällt.

\section{Grundlegende Konzepte}
Da die Kommunikationseinrichtungen der Hardware Byte-orientiert sind, wurde das
Protokoll ebenfalls Byte-orientiert aufgebaut. Im Gegensatz zu dem Vorgänger-Protokoll
werden also keine Zeichenketten benutzt, um die Informationen zu repräsentieren, sondern
nur der Wert der einzelnen Bytes.\\
Ein Befehl ist eine Folge von n Bytes, wobei n mindestens 1 und maximal die im Code
eingestellte Größe der Befehlsstruktur entspricht. Von den eingebauten Befehlen besitzt
keiner eine maximale Länge von mehr als 9 Bytes. Das erste Byte eines Befehls hat eine
besondere Bedeutung. Dieses Byte wird Kommando-Byte genannt. Dieses Kommando-Byte
ist in zwei Teile geteilt. Der erste Teil umfasst die niederwertigsten 4 Bit und
wird Befehlscode genannt. Dieser Befehlscode spezifiziert die Art des Befehls.
Der zweite Teil beinhaltet die höchstwertigen 4 Bit. In ihm werden Optionen
angegeben, die den im Befehlscode angegebenen Befehl modifizieren. Die Kombination
von Befehlscode und Optionen legt auch die Länge des Befehls fest. Alle auf dem
Kommando-Byte folgende Bytes sind Parameter, wie z.B. die Geschwindigkeit der Räder.\\
Da 4 Bit für Befehlscodes zur Verfügung stehen, sind 16 verschiedene Befehle möglich.
Sechs Befehle wurden im Zuge dieser Arbeit implementiert, das würde noch Platz für
10 weitere Befehle lassen. Das Protokoll sollte allerdings noch mehr Freiheiten
für zukünftige Erweiterungen bieten, deswegen wurde einer der Befehlscodes
reserviert. Dieser reservierte Befehlscode und dessen Behandlung im Code, ermöglichen
es, dass es mehr als ein Kommando-Byte gibt. Damit ist es möglich Befehle einfach 
hinzuzufügen, die nicht auf das ''ein Kommando-Byte, viele Parameter''-Schema passen.\\
Wenn Parameter übertragen werden müssen, die mehr als ein Byte benötigen, wird zuerst
das höherwertigste Byte übertragen. Danach absteigend nach der Wertigkeit die anderen
Bytes.\\
Durch die volle Ausnutzung der Bytes gibt es kein Protokoll-spezifisches STOP- oder START-Byte.
Das bedeutet, dass das Protokoll sich auf die Flusskontrolle der zugrunde liegende Hardware
verlässt.

\section{Eingebaute Befehle}
Sechs Befehle und die Infrastruktur für den reservierten Befehlscode wurden implementiert.
Im nachfolgenden werden die Befehle einzeln vorgestellt. Dabei wurd der Befehlscode bei
jeden Befehl in hexadezimaler Schreibweise als ganzes Byte dargestellt.

\subsection{Extended Instruction}
Dieser Befehl ist ein Platzhalter für zukünftige Befehle, die mehr als ein Kommandobyte benötigen,
oder wenn alle normalen Befehlscodes bereits vergeben sind. Der Befehlscode ist 0x00.\\
Im Code werden Befehle dieser Art mithilfe von besonderen Funktionen behandelt. Die 
parser\_\-extended\_\-order\_\-complete-Funktion ist ein Beispiel hierfür. Diese
besonderen Versionen von Funktionen gibt es allerdings nur im Parser, danach wird eine Extended
Instruction genauso behandelt wie ein normaler Befehl. Diese zusätzlichen Funktionen
waren nötig, da sich diese Befehle im Aufbau sehr von normalen Befehlen unterscheiden.

\subsection{Control}
Mithilfe dieses Befehls kann das System gesteuert werden. Darunter fallen Aufgaben wie das
Reseten des gesamten Programs, das Anhalten der Befehlsausführung und die Manipulation
der Befehls-Warteschlange. Der Befehlscode für diesen Befehl ist 0x01. Außerdem ist der Control-Befehl
ein priorisierter Befehl, d.h. er wird bei dem nächsten Hauptschleifendurchlauf ausgeführt und nicht
an die Warteschlange angereiht.\\
Die in Tabelle \ref{protocol_control} beschriebene Bitmaske wird mit dem Befehlscode verundet
und ergibt das Kommandobyte. Bei manchen Befehlen können mehrere Optionen zur gleichen Zeit
gewählt werden, dies ist hier nicht der Fall. Die Optionen dieses Befehlsschließen sich
gegenseitig aus. Die Länge der Befehle mit diesem Befehlscode beläuft sich auf 1 Byte, also
lediglich das Kommandobyte muss gesendet werden.
\begin{table}[htb]
\begin{center}
	\begin{tabularx}{\linewidth}{|l|l|X|}
		\hline
		\textbf{Option} & \textbf{Bitmaske} & \textbf{Beschreibung} \\
		\hline
		\hline
		Reset 			& 0x10 				& Resetet die Hardware \\
		\hline
		Stop Queue		& 0x20				& Der Aktuelle Befehl wird verworfen. Queue wird angehalten, aber nicht verworfen. \\
		\hline
		Continue Queue	& 0x30				& Der nächste Befehl in der Queue wird ausgeführt. \\
		\hline
		Clear Queue		& 0x40				& Die Warteschlange wird gelöscht. \\
		\hline
		Stop Drive		& 0x50				& Befehl wird angehalten, Fahrzeug geht zum aktiven Bremsen über. \\
		\hline
	\end{tabularx}
	\caption{\label{protocol_control} Optionen des Control-Befehls}
\end{center}
\end{table}

\subsection{Query}
Manchmal ist es für die kontrollierende Praktikumsplatine wichtig verschiedene Laufzeitwerte
der Motorplatine zu kennen. Dieser Befehl, dem der Befehlscode 0x02 zugeordnet ist, ermöglicht
es die Geschwindigkeit der Räder, die Anzahl der Befehle in der Warteschlange und den aktuellen
Befehl abzufragen. Die Optionen schließen sich gegenseitig aus und die Länge des Befehls beträgt
immer 1 Byte. Der Befehl ist wie der Control-Befehl ein priorisierter Befehl.\\
Wenn dieser Befehl an die Motorplatine gesendet wurde und kurz darauf versucht wird über den
I2C-Bus das Ergebniss zu lesen, kann es vorkommen, dass das Ergebniss noch nicht bereit ist.
Dies ist insbesondere der Fall, wenn der aktuelle Befehl angefordert wurde. In diesem Fall muss
die Lese-Operation nochmal gestartet werden.\\
Damit die Praktikumsplatine etwas mit dem aktuellen Befehl, den die Motorplatine zurückgibt,
anfangen kann, muss die Länge des Befehls mit übertragen werden. Deswegen ist es nötig,
dass die Praktikumsplatinen zwei Lese-Operationen durchführt. Die erste hat eine Länge
von einem Byte und spezifiziert wie lang die zweite Antwort in Bytes ist.

\begin{table}[htb]
\begin{center}
	\begin{tabularx}{\linewidth}{|l|l|X|}
		\hline
		\textbf{Bitmaske} & \textbf{Beschreibung} & \textbf{Antwortlänge} \\
		\hline
		\hline
		0x10 				& Geschwindigkeit des linken Rades & 1 \\
		\hline
		0x20				& Geschwindigkeit des rechten Rades & 1 \\
		\hline
		0x30				& Anzahl der Befehl in der Queue & 1 \\
		\hline
		0x40				& Aktueller Befehl & 2 - 16 \\
		\hline
	\end{tabularx}
	\caption{\label{protocol_queue} Optionen des Queue-Befehls}
\end{center}
\end{table}
\subsection{Drive}
\subsection{Advanced Drive}
\subsection{SetPID}
\subsection{Option}
