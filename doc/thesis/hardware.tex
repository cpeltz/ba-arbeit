\chapter{Überblick über die Hardware}

Die Motorplatine wurde im Zuge der Studienarbeit von Timo Klingeberg \cite{STUD_TIMO}
entwickelt.
\begin{figure}[htb]
 \centering
 \scalebox{0.5}{\includegraphics{pictures/board.png}}
 \caption{\label{board}Die Motorplatine}
\end{figure}
\section{Mikrocontroller}
Das Herzstück der Platine bildet ein Mikrocontroller der Firma Atmel.
Es handelt sich hierbei um einen ATMEGA2561\cite{ATMEGA_MANUAL}, der 256 KiB Speicher für
Programme (Flash) hat, sowie 8 KiB Speicher für Variablen (SRAM). Der maximale Takt für
diesen Mikrocontroller liegt bei 16 MHz, der auch ausgenutzt wird. Das bedeutet, dass
ein Takt 62,5 ns benötigt. Da der Großteil der Instruktionen des Mikrocontrollers nur
einen Takt benötigt, kann dieser theoretisch 16 MIPS an Leistung erreichen. Durch die Benutzung
von Funktionen, Pin-IO und bedingten Sprüngen bleibt dies aber nur eine theoretische Zahl.
Vorteilhafterweise unterstützt der Controller das ISP (In-System Programming). Dies war bei der
Entwicklung der Betriebssoftware von großem Nutzen, da hierdurch schnell und relativ
unkompliziert neue Versionen auf die Platine überspielt werden konnten.\\
Der Controller verfügt über 6 Timer, zwei von diesen sind nur 8-Bit, die anderen vier allerdings
haben 16-Bit, was bei dem gegebenen Takt von 16 MHz einen maximalen Timer-Intervall von 262 ms
entspricht. Zusätzlich kann der Controller sechs Puls-Weiten-Modulationen betreiben, von
denen zwei für die Benutzung der Servomotoren, die die Räder antreiben, benutzt werden.
\section{Ein-/Ausgabemöglichkeiten zur Praktikumsplatine}
Die Motorplatine verfügt sowohl über einen UART-Port, als auch einen I2C-Bus, über die
die Praktikumsplatine mit der Motorplatine kommunizieren kann. Der UART-Port wird außerdem
für die Ausgabe von Debug-Informationen benutzt, falls dies aktiviert wird. Im allgemeinen
Fall wird allerdings nur der I2C-Bus zur Kommunikation zwischen den beiden Platinen verwendet.
Da der Mikrocontroller über eingebaute Hardwarelogiken sowohl für den I2C-Bus als auch für
den UART-Port verfügt, hält sich der administrative Aufwand für die Kommunikation sehr in
Grenzen. Es können lediglich Interrupt Service Routinen (ISR) für diese zur Verfügung gestellt
werden, die es dann ermöglichen schnell auf Situationen zu reagieren und ohne die Gefahr
einzugehen Informationen zu verlieren.
\section{Interne Ein-/Ausgabe-Ports}
Zusätzlich zu den externen Kommunikationsmöglichkeiten, besitzt der ATMEGA2561 54
programmierbare IO Kanäle, die in Ports mit je 8 Kanälen zusammenfasst werden. 16 von diesen
Kanälen sind für die Steuerung der Servomotoren zuständig, die die Räder antreiben, sowie bieten
Rückmeldung von den Hall-Sensoren, die an den Motoren befestigt sind, um bei Bewegung der Räder
Interrupts auslösen zu können und aus den Informationen der Sensoren schließen zu können, in welche
Richtung die Räder sich drehen, was wiederum notwendig ist um nur eine bestimmte Strecke zu fahren.
Für jede Umdrehung eines Rades werden 360, also für jedes Grad einer, Interrupts ausgelöst, je
nachdem wie groß der Durchmesser des Rades ist, ist eine Steckenauflösung  von wenigen Millimetern
möglich.
